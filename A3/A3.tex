\documentclass[12pt]{article}

\usepackage[a4paper,margin=1in]{geometry}

\usepackage[T1]{fontenc}
\usepackage{epsfig,epsf,psfrag,array,amsmath,amssymb,verbatim,setspace,bm}
\usepackage{url,comment,enumerate,graphicx,graphics,wrapfig}
\usepackage[usenames,dvipsnames,svgnameS,table]{xcolor}
\usepackage[boxed,linesnumbered]{algorithm2e}
\usepackage{pdfpages,framed,tcolorbox}
\usepackage{hyperref}
\usepackage{enumitem}
\usepackage{mathtools}
\DeclarePairedDelimiter\ceil{\lceil}{\rceil}
\DeclarePairedDelimiter\floor{\lfloor}{\rfloor}
\setlength\parindent{0pt}

\title{\vspace{-6mm}{ Assignment-3}}
\author{Anannya Mathur 2023SIY7565}
\date{}
\setcounter{secnumdepth}{0}

\begin{document}
\maketitle
\noindent
\vspace{-10mm}
\section{Question 1}

Let C= k-Min-Cut = minumum number of edges whose removal
will partition the graph G into k separate components. \\
\\
The random edge contraction algorithm of Karger and Stein
terminates when there are two vertices left. Instead, the 
algorithm can be made to stop when there are k vertices 
remaining. The success probability can be computed as 
follows: \\
Let |C| = $\lambda$. Let $X \in V $ such that $|X|=k-1.$ We 
can observe that upon the removal of all those edges whose both 
the endpoints lie in X and the edges whose one endpoint is in X 
and the other in $X^C$, we create k-1 connected components 
consiting of just one vertex, which correspond to vertices 
that are a part of set X, while
the vertices in $X^C$ will have created $\ge 1$ connected 
components. Thus, the set of edges we removed corresponds to
a k-cut. For any set X, the size of the set that contains edges 
$e \in E$ such that none of the endpoints of e lie in X 
$\le m- \lambda $. Number of ways we can create a set X=
$ {n} \choose {k-1}$.For and edge e $\in E$, the size of a set
such that none of the endpoints of e lie in X = ${n-2} 
\choose {k-1}$. $\therefore {{n-2} \choose {k-1}}m \le
{{n} \choose {k-1}}(m- \lambda)$
\\ 
$\implies \frac{{{n-2} 
\choose {k-1}}}{{{n} \choose {k-1}}} \le 1- \frac{\lambda}{m}=$ 
Probability that a random edge selection is not in C. \\
\\
$\therefore$ Prob(preserving k-min-cut until k vertices left)
$\ge \frac{{{n-2} 
\choose {k-1}}}{{{n} \choose {k-1}}} \frac{{{n-3} 
\choose {k-1}}}{{{n-1} \choose {k-1}}}....\frac{{{k} 
\choose {k-1}}}{{{k+2} \choose {k-1}}}\frac{{{k-1} 
\choose {k-1}}}{{{k+1} \choose {k-1}}}$= $\frac{{{k} 
\choose {k-1}}}{{{n} \choose {k-1}}}\frac{{{k-1} 
\choose {k-1}}}{{{n-1} \choose {k-1}}} = \frac{{{k} 
\choose {k-1}}}{{{n} \choose {k-1}}}\frac{{{k-1} 
\choose {k-1}}}{{{n-1} \choose {k-1}}}= \frac{k}{{{n} \choose {k-1}}}\frac{1}{{{n-1} \choose {k-1}}}
\ge \frac{2}{n^{2k-2}}.$ 
\\ \\
If we run the algorithm $4n^{2k-2}logn$ times, Prob(k-min-cut
fails to survive n-k random edge contractions) $\le (1-\frac{2}{n^{2k-2}})^{4n^{2k-2}logn}
\le \frac {1}{n^2}.$ Hence, with a success probability of 
$1- \frac{1}{n^2}, $ a global k-min-cut can be computed in 
$O((n-k)n^{2k-1}logn).$ \\

Therefore, a 3 min-cut can be computed by plugging in the value of 
k as 3. With a success probability of 
$1- \frac{1}{n^2}, $ a global 3-min-cut can be computed in 
$O(n^6logn).$

\newpage

\section{Question 2}

Given G and a matching M, the Edmonds algorithm finds odd 
length cycles and shrinks them into a supernode. Therefore, 
if the algorithm finds an alternating path, beginning with an 
unmatched edge and ending with an unmatched edge, from a supernode 
to itself, a new supernode can be formed as the path forms an odd 
length cycle. Each supernode $\in V_{odd}(T)$ has size one 
while every supernode $\in V_{even}(T)$ has odd size. We define 
every supernode by its base vertex. For supernodes having just 
one vertex, the base vertex is the vertex itself. If an alternating path 
begins at supernode $W_o$ and cycles back to itself, the new supernode W
will have its base defined as the base of $W_{o}$. It can 
be noticed that if the base vertex of a supernode is matched, the 
supernode is matched too. Instead of 
contracting odd-length cycles, we store for every vertex the base vertex of 
supernode it belongs to and the edge $e_{i}$ used to enter the 
supernode. In case multiple supernodes are merge into one supernode, S, we 
store a list of sub-supernodes, S1, S2,....Si, so that we can change the 
status of edges in the event of an augmenting path discovery. 
If we encounter an unmatched edge e=(u,v) provided u, v do not 
belong to the same supernode, such that u, part of a 
supernode say $W_i$ can be reached 
from a free supernode via an alternating path P=(e1,e2,...,ei), where
e1 is unmatched while ei is matched, we consider the following cases for v:
1. v is a part of free supernode $W_j$. 2. v is a part of a matched supernode.
Since for every vertex, we have the information stored regarding the supernode 
it belongs to and the edge e used to enter the supernode, we can backtrack using
those edges along the two paths, one beginning with the supernode $W_j$, which contains
the vertex v and the other beginning with the supernode $W_i$, which contains
the vertex u. If we find the two paths terminating at 
unmatched supernodes A $\ne$ B, then we have found an 
augmenting path. In case, the paths converge at the same supernode, say C, 
a new supernode should be formed, which contains the list of 
all the sub-supernodes contained within it, which can easily be 
found as they lie on the path connecting $W_i$ and C and $W_j$ and C. \\
During this 
traversal in search of a new supernode or an augmenting path, every edge e=(u,v) that we encounter 
along the way should be treated in a similar fashion given that it satifies the mentioned condition. 
In case u and v belong to the same supernode, we discard the edge. The algorithm terminates 
when it can no longer find a valid edge. 
\\
For every edge that fulfills the condition, it will take at most 
O(n) number of supernodes to traverse. It can be noticed that the supernodes 
are disjoint in nature. Since we are using Union-Find 
data structure for both concatenating multiple supernodes into one and 
finding the supernode a vertex is present in, it will take O(mlogn) time 
for m operations on a disjoint set of n objects. Number of augmentations 
$\le \frac{n}{2}$. Therefore, the implementation will take 
O(mnlogn). \\ \\

In an M-alternating tree T under some maximum matching M in G,
a Tutte-Berge maximiser is the set of vertices in G lying 
in $V_{odd}(T_{final})$, where $T_{final}$ is the final tree
obtained post all valid cycle contractions. These vertices represent the vertices in 
$T_{final}$ that can be visited by a 
free vertex via a path that starts with an unmatched edge 
and ends with an unmatched edge, that is, an odd-length path. Therefore, 
the above solution can be further extended to output the base vertices of supernodes in the 
final graph that fulfill 
the condition of being reachable from a free supernode via an odd-length path.
Such supernodes are guaranteed to be singleton sets consisting of 
just one vertex. 

\newpage 

\section{Question 3}

Let us consider a bipartite graph G= (X,Y,E). There can exist a path 
from X to Y or Y to X, therefore, if there does exist a cycle in G, it 
has to be of even length. \\
Let us consider that G has two edge 
disjoint perfect-matching, say M and M'. $M \cap M' = \emptyset $, which 
implies that for any vertex x $\in X$, there exists an edge x,a under M and 
x,b under M', where a and b $\in Y$. Therefore, G is 2-regular. If we pick 
any vertex in G and follow one of the two edges to reach another vertex and keep 
traversing one of the two outgoing edges of every vertex we encounter, we 
will eventually reach the original vertex, thus, discovering one of the cycles.
We can do the same for all the unvisited vertices to expose the other cycles 
present in G. These cycles are disjoint from the rest and their union 
covers all the vertices in G. Therefore, if G has 2-edge disjoint 
perfect matching, it has a disjoint cycle cover. \\
If G has a disjoint cycle cover, every vertex must have 2 edges connecting it 
to the other vertex so that we can follow one of the two edges to reach the other 
vertex and keep traversing and we are bound to cycle back to the 
original vertex. Therefore, G is 2-regular $\implies$ G has two edge 
disjoint perfect-matching. \\ \\

2-edge disjoint perfect matching can be found by a single max-flow computation
as follows: \\
We can introduce two vertices s and t. We connect s with all the 
vertices in X and assign the capacity of these edges as 2. Similarly,
we connect all the vertices in Y to t with the capacity of these edges as 2.
To all the edges connecting vertices in X and Y in G, we assign the capacity 
as 1. We compute the (s,t)-min cut for the updated graph and if the 
computed max flow value $\lambda$=
2|X|=2|Y|, the original bipartite graph G indeed has 2-edge disjoint perfect matching, 
otherwise, it does not.

\end{document} 