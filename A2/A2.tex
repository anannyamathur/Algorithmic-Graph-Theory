\documentclass[12pt]{article}

\usepackage[a4paper,margin=1in]{geometry}

\usepackage[T1]{fontenc}
\usepackage{epsfig,epsf,psfrag,array,amsmath,amssymb,verbatim,setspace,bm}
\usepackage{url,comment,enumerate,graphicx,graphics,wrapfig}
\usepackage[usenames,dvipsnames,svgnameS,table]{xcolor}
\usepackage[boxed,linesnumbered]{algorithm2e}
\usepackage{pdfpages,framed,tcolorbox}
\usepackage{hyperref}
\usepackage{enumitem}
\usepackage{mathtools}
\DeclarePairedDelimiter\ceil{\lceil}{\rceil}
\DeclarePairedDelimiter\floor{\lfloor}{\rfloor}
\setlength\parindent{0pt}

\title{\vspace{-6mm}{ Assignment-2}}
\author{Anannya Mathur 2023SIY7565}
\date{}
\setcounter{secnumdepth}{0}

\begin{document}
\maketitle
\noindent
\vspace{-10mm}
\section{Question 1}
\subsection{Question (a):}

$ Prob( a,b \in S_{i}, F \cap S_{i}= \emptyset) = (\frac {1} {k-1})^2(1-\frac {1}{k-1})^{k-1} \ge 
\frac {1}{4(k-1)^2} \\
Prob(\forall i, S_{i} \: \text {does not separate a,b from F })
= \prod_{1\le i\le r} (1-Prob(a,b \in S_{i}, F \cap S_{i}=\emptyset))
\le (1- \frac {1}{4(k-1)^2})^r \le \frac {1}{n^t}, \text 
{where r=} 4t(k-1)^2\log_{e}(n). \\
Prob(\exists ((a,b),F) \: \text {satisfying (a,b),F are not
separated by } S_{i}) \le \sum_{((a,b),F)\in V^2 \times V^{k-1}} 
\frac{1}{n^t} \le \frac {n^{k+1}} {n^t} \\
\therefore \text {t should be k+2} \\
\therefore r= 4(k+2)(k-1)^2\log_{e}n
$ 


\subsection{Question (b):}
$
\text {Every spanning forest } T_{i} \text { will have }
\le |V_{i}| \: edges. \\
\text{Probability of a vertex appearing=} \frac {1}{k-1}. \\
\text {The expected number of edges in H } \le 
\sum_{i=1}^{r} (\frac {1} {k-1})^{|V_{i}|}|V_{i}|
\le \sum_{i=1}^{r} \frac {1} {k-1}n \\ = 
\frac {n} {k-1} 4(k+2)(k-1)^2\log_{e}n \approx O(nk^2logn)\\ \\
Prob(a\in G, a\notin H )= (1- \frac{1}{k-1})^r \le
\frac {1}{n^{4(k+2)(k-1)}} \\
\implies \text {G and H share the same set of vertices with
high probability.} \\
\text {Given: G is k-vertex connected.} \\
\text {To prove: H is k-vertex connectivity preserver of G
with high probability.}\\
\text {Suppose H is not k-vertex connected.} \\ \therefore 
\exists $ a vertex cut of size at most k-1 upon whose
removal, H gets disconnected. We call this vertex cut F.
Since G is given to be k-vertex connected, removal of F 
should not disconnect G $\implies$ there must exist a path
containing, say vertices a and b, in G-F. With a probability
$\ge 1- \frac{1}{n},$ there must exist some $S_{i}$ that
contains a,b but $F \cap S_{i} = \emptyset. \: \therefore$
No vertex from F is contained in $S_{i}. \: \therefore$ There 
must exist a path connecting a and b in $T_{i}$ that excludes
vertices present in F. Therefore, there must exist a path
connecting a and b in H-F as well since G and H share the same
set of vertices with high probability. Thus, removal of F 
does not disconnect H and thus, our earlier claim of H not
being k-vertex connected stands false. This proves that H is
indeed a k-vertex connectivity preserver of G. 

\pagebreak

\section {Question 2:}

\subsection{Question a:}

P(degree of vertex=d)= $ {{n-1} \choose {d} } p^d (1-p)^{n-1-d} $ \\
$\therefore$ P(node v is isolated) = $(1-p)^{n-1}$ \\
P(i, n-i)= P(a set of nodes is disconnected from n-i nodes
where $i \in [1,n/2]$)= \\ ${n \choose i}(1-p)^{i(n-i)},$ as
i(n-i) edges will have to be absent to disconnect the partition 
(i, n-i). \\
$\therefore$ P(G is disconnected) $\le \sum_{i=1}^{n/2} 
{n \choose i}(1-p)^{i(n-i)}=\sum_{i=1}^{n/2} 
{n \choose i}(1- \frac{5\log n}{n})^{i(n-i)} \le
\sum_{i=1}^{n/2} {n \choose i} \frac {1}{n^{\frac{5i(n-i)}{n}}}
\le \sum_{i=1}^{n/2} \frac{n^i}{i!} \frac {1}{n^{\frac{5i(n-i)}{n}}} 
\le \sum_{i=1}^{n/2} \frac{n^i}{n^{\frac{5i}{2}}}
= \sum_{i=1}^{n/2} \frac{1}{n^{\frac{3i}{2}}} \le \frac{n}{n^3}=
\frac{1}{n^2} \le \frac{1}{n} \implies
$ G is connected with probability at least $1-\frac{1}{n}$.

\subsection{Question b:}

Graph G should be a k-edge connected graph. $\therefore$ G 
should remain connected if k-1 edges are removed but should get 
disconnected if k edges are removed. Suppose G splits into
(i, n-i-k) when k edges are removed but remains connected if 
k-1 edges are removed instead. \\
Prob(G is a k-edge connected graph) $\le \sum_{t=k}^{ik}
{n \choose k}{n-k \choose i}{ik \choose t}p^t (1-p)^{i(n-i)-t},$ 
where ${n \choose k}$ denotes choosing k edges out of n;
${n-k \choose i}$ denotes choosing i nodes out of the remaining 
n-k nodes upon k edges removal; ${ik \choose t}$ denotes
choosing edges that keep the graph connected when k-1 edges
are removed. \\
$\therefore$ P(G is k-edge connected) $\le \sum_{t=k}^{ik}
\frac {n^k}{k!} \frac{(n-k)^i}{i!}\frac{(ik)^t}{t!}p^t(1-p)^{i(n-i)-t}
\le \sum_{t=k}^{ik}
\frac {n^k}{k!} \frac{(n-k)^i}{i!}\frac{(ik)^t}{t!}p^t(1-p)^{i(n-i)-ik}
\le \sum_{t=k}^{ik}
\frac {n^k}{k!} \frac{(n-k)^i}{i!}\frac{(ik)^t}{t!}(\frac{cklogn}{n})^tn^{-(cki(n-i-k))/n}
\le \frac {n^2}{n^c} \le \frac {1}{n}. \\
\implies$ GH Tree of G has edge weights of at least k with a 
probability at least $1-\frac{1}{n}.$

\subsection{Question c:}

\includegraphics{img1.png} \\
\includegraphics{img2.png} \\
\includegraphics{img3.png}\\
\includegraphics{img4.png}\\

Adding an edge will reduce the number of connected components to
create a larger connected component, as evident in the graphs.
Upon increasing the probability of an edge getting added,
the number of connected component decreases, while the size of
largest connected component increases. When p is of the order
O($\frac {\lambda logn} {n}$), where $\lambda > 1,$ the graph 
G(n,p) is connected with a probability $\ge 1- \frac{1}{n}, $
as shown in Q2 part a and b. As evident in the plots, the 
size and number of connected components change dramatically
when p reaches order $O(\frac {logn} {n}) \approx log(40)/40=0.04.$ 
Once p crosses this threshold, the size of largest connected component
increases while the number of connected components decreases as 
the graph begins to get connected. 

\end{document}

 